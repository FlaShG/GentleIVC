\documentclass[twocolumn]{article}

\usepackage[german]{babel}
\usepackage[utf8]{inputenc}
\usepackage[margin=1.25in]{geometry}
\usepackage{fancyhdr}

\author{Sascha Graeff, Max Menzel, Mario Mohr}
\title{Alice \\ \small Projektdokumentation \\ Interactive Visual Computing \\ WiSe 2012/13}

    \makeatletter
    \renewcommand\section{\@startsection{section}{1}{\z@}%
                                      {-3.5ex \@plus -1ex \@minus -.2ex}%
                                      {2.3ex \@plus.2ex}%
                                      {\normalfont\large\bfseries}}
    \makeatother

\pagestyle{fancy}
\fancyhead{}
\fancyhead[LE, LO]{\footnotesize \leftmark}
\fancyhead[RE, RO]{\footnotesize \rightmark}


\begin{document}


\maketitle

\abstract{In diesem Paper beschreiben wir die in unserem Kurzfilm 'Alice' verwendeten Konzepte und Werkzeuge.}


\section{Einführung}

\section{Allgemeines}
\subsection{Tiefenunschärfe}
\subsection{Alice}
\subsection{Der Gentleman und gentle.inc}

\section{Szenen 1-10: Der Pilzwald}
\subsection{Die Pilzwald-Umgebung}
\subsection{Die ChesSir-Cat}

\section{Szenen 11-20: Die Taschenuhr}
\subsection{Die Taschenuhr}
\subsection{Die Teaparty}
\subsection{Die Riesen-Pilzwald-Umgebung}

\end{document}